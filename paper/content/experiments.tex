\section{Experiments}\label{sec:experiments}

\subsection{Environmental Setup}\label{ch:ev}
For our Experiments we are bounded by the Environments given from Kaggle \cite{noauthor_kaggle_nodate} and Colab \cite{noauthor_google_nodate}, 
which makes quite complex to find the optimal hyperparameter. Also, there is no possibility to Cache or Precalculate all the Datapoints of
Training and Validation without running in to RAM issues. We decided to shift the bottlenecks either in parallel loading the
Data and/or running in exceeding GPU RAM due to the amount of parameters.

\vspace{0.3cm}

\begin{minipage}{0.5\textwidth}
    Colab's Environment:
    \begin{itemize}
        \item CPU: Intel(R) Xeon(R) @ 2.00GHz
        \item Number of available Cores: 2
        \item System Ram: 12 GB
        \item GPU: Nvidia Tesla T4
        \item GPU Ram: 15 GB
        \item Time limit: 3-4 Hours a Day/Session
    \end{itemize}
\end{minipage}
\begin{minipage}{0.5\textwidth}
    Kaggle's Environment:
    \begin{itemize}
        \item CPU: Intel(R) Xeon(R) @ 2.00GHz
        \item Number of available Cores: 4
        \item System Ram: 32 GB
        \item GPU: Nvidia Tesla P100
        \item GPU Ram: 16 GB
        \item Time limit: 30 Hour a Week with 9h each Session
    \end{itemize}
\end{minipage}

\vspace{0.3cm}

The general handling of the loops is based on Pytorch \cite{noauthor_pytorch_nodate}, Lightning AI \cite{noauthor_lightning_nodate}
implementation and for loading the pre-trained BERT Model we are using Huggingface (transformer library) \cite{noauthor_hugging_2023}.
Therefore, we can focus the Experiments on implementing strategies to enhance BERT's overall performance tested on four different Seeds.











\subsection{Learning Rate}

In the evaluation of the Baseline BertModelForSentenceClassification, different Learning-Rate 
were tested, accompanied by variations in token size and the implementation of mixed precision.
The tested Range of Learning-Rate (see Table~\ref{tab:my-lr}) going from $3e-6$ to $7e-6$. All above or
below leading that the Model does not learn well or resulting in a rectangle graph, indicating
that the gradients are too high or low. The best Learning Rate, based on the Seeds is $6e-6$ with
$0.640 \pm 0.021$.

However, it was observed that altering the token size and introducing mixed precision led to 
the destruction of tensors within the model leading to re pre-train it whitin the Environment
this is impossible. 
This suggests potential challenges and limitations associated with modifying these parameters, 
emphasizing the need for careful consideration and experimentation when adjusting 
learning rates and employing mixed precision in the context of sentence classification 
tasks using the BERT model.

\begin{table}[h]
    \centering
    \caption{F1 Values of the Baseline BertModelForSentenceClassification of the Last (50th) Epoch
             for different Learning-Rates.}
    \label{tab:my-lr}
    \begin{tabular}{|c||cccc|c|}
    \hline
    \multicolumn{1}{|l||}{\multirow{2}{*}{Learning Rate}} & \multicolumn{4}{c|}{Seed}                                                                    & \multicolumn{1}{c|}{\multirow{2}{*}{mean $\pm$ std}} \\ \cline{2-5}
    \multicolumn{1}{|l||}{}                               & \multicolumn{1}{c|}{0}     & \multicolumn{1}{c|}{42}    & \multicolumn{1}{c|}{1998}  & 1M    & \multicolumn{1}{c|}{}                                \\ \hline\hline
    3e-6                                                 & \multicolumn{1}{c|}{0.637} & \multicolumn{1}{c|}{0.554} & \multicolumn{1}{c|}{0.623} & 0.652 & 0.614 $\pm$ 0.034                           \\ \hline
    4e-6                                                 & \multicolumn{1}{c|}{0.589} & \multicolumn{1}{c|}{0.614} & \multicolumn{1}{c|}{0.649} & 0.615 & 0.616 $\pm$ 0.019                                    \\ \hline
    5e-6                                                 & \multicolumn{1}{c|}{0.636} & \multicolumn{1}{c|}{0.640} & \multicolumn{1}{c|}{0.638} & 0.657 & 0.630 $\pm$ 0.026                                    \\ \hline
    6e-6                                                 & \multicolumn{1}{c|}{0.661} & \multicolumn{1}{c|}{0.654} & \multicolumn{1}{c|}{0.602} & 0.647 & \textbf{0.640 $\pm$ 0.021}                                    \\ \hline
    7e-6                                                 & \multicolumn{1}{c|}{0.644} & \multicolumn{1}{c|}{0.563} & \multicolumn{1}{c|}{0.657} & 0.655 & 0.621 $\pm$ 0.040                                    \\ \hline
    \end{tabular}
\end{table}
\vspace{-0.5cm}











\subsection{Permutation}
\begin{equation}\label{eq:permutate}
    \resizebox{0.91\hsize}{!}{%
     $Sentence = [CLS] \; \textcolor{red}{stm}\;
                 [SEP] \; \textcolor{green}{sec}\;
                 [SEP] \; \textcolor{blue}{\pi(p_1), \pi(p_2), ... }\;
                 [SEP] \; \textcolor{blue}{\pi(s_1), \pi(s_2), ... }\;
                 [SEP]$      
    }
\end{equation}

As described above changing Token size does lead to destroying the embeddings due to newly
initializing and the fact that the maximum token size is 512, we come up with the idea
of permuting the Primary and Secondary Trials (see Equation~\ref{eq:permutate}), which can be seen as list of items.
The Target to analyze is if the order of the elements is relevant for deciding the Prediction, 
wich should not be relevant due to BERT's Architecture, and
if information after the cutpoint has interesting and relevant facts, which influences
the decision making process. With no permutation the 
Sentence is only 512 tokens long with the first-come-first-serve principle. Permutation on
indicates that the order is not significant.

\begin{table}[!t]
    \centering
    \caption{F1 Values of the Permutation trial on the Baseline Model from the last (50th) Epoch}
    \label{tab:my-permutation}
    \begin{tabular}{|c|c||cccc|c|}
    \hline
    \multirow{2}{*}{Learning Rate} & \multirow{2}{*}{Permutation} & \multicolumn{4}{c|}{Seed}                                                                    & \multirow{2}{*}{mean $\pm$ std} \\ \cline{3-6}
                                   &                              & \multicolumn{1}{c|}{0}     & \multicolumn{1}{c|}{42}    & \multicolumn{1}{c|}{1998}  & 1M    &                                 \\ \hline \hline
    \multirow{2}{*}{5e-6}          & No                           & \multicolumn{1}{c|}{0.636} & \multicolumn{1}{c|}{0.640} & \multicolumn{1}{c|}{0.638} & 0.657 & 0.630 $\pm$ 0.026               \\ \cline{2-7} 
                                   & Yes                          & \multicolumn{1}{c|}{0.633} & \multicolumn{1}{c|}{0.652} & \multicolumn{1}{c|}{0.620} & 0.643 & \textbf{0.643 $\pm$ 0.016}      \\ \hline
    \end{tabular}
\end{table}

The Results (see Table~\ref{tab:my-permutation}) showed that there can be a slight Performance boost and the standard deviation is lower
indicating that the fluctuation of different seeds in not that harsh. But, the Runtime cost of permuting the
sentences every Batch is for the later Experiments too much. Therefore, we skip it to stay cost-efficient.














\subsection{Dataset Expansion}\label{ch:ds}

\begin{table}[!b]
    \centering
    \caption{F1 Values of the Dataset Expansion Test of Last (50th) Epoch for the different Startegies and Datasets, which are expanding SemEval. The Values are the SemEval only Validation.}
    \label{tab:my-dsexp}
    \resizebox{\textwidth}{!}{%
    \begin{tabular}{|c||cc|cccc|c|}
    \hline
    \multirow{2}{*}{Dataset + SemEval} & \multicolumn{2}{c|}{Strategy}        & \multicolumn{4}{c|}{Seed}                                                                    & \multirow{2}{*}{mean $\pm$ std} \\ \cline{2-7}
                                       & \multicolumn{1}{c|}{CombDL} & CombDS & \multicolumn{1}{c|}{0}     & \multicolumn{1}{c|}{42}    & \multicolumn{1}{c|}{1998}  & 1M    &                                 \\ \hline\hline
    Baseline                           & \multicolumn{1}{c|}{}       &        & \multicolumn{1}{c|}{0.636} & \multicolumn{1}{c|}{0.640} & \multicolumn{1}{c|}{0.638} & 0.657 & 0.630 $\pm$ 0.026               \\ \hline\hline
    
    \multirow{2}{*}{SNLI}              & \multicolumn{1}{c|}{\chm}   &        & \multicolumn{1}{c|}{0.597} & \multicolumn{1}{c|}{0.613} & \multicolumn{1}{c|}{0.685} & 0.573 & 0.617 $\pm$ 0.042               \\ \cline{2-8} 
                                       & \multicolumn{1}{c|}{}       & \chm   & \multicolumn{1}{c|}{0.619} & \multicolumn{1}{c|}{0.669} & \multicolumn{1}{c|}{0.648} & 0.634 & 0.643 $\pm$ 0.019               \\ \hline
    
    \multirow{2}{*}{HEALTHVER}         & \multicolumn{1}{c|}{\chm}   &        & \multicolumn{1}{c|}{0.634} & \multicolumn{1}{c|}{0.614} & \multicolumn{1}{c|}{0.622} & 0.610 & 0.620 $\pm$ 0.009               \\ \cline{2-8} 
                                       & \multicolumn{1}{c|}{}       & \chm   & \multicolumn{1}{c|}{0.637} & \multicolumn{1}{c|}{0.661} & \multicolumn{1}{c|}{0.613} & 0.631 & 0.635 $\pm$ 0.017               \\ \hline
    
    \multirow{2}{*}{SCIFACT}           & \multicolumn{1}{c|}{\chm}   &        & \multicolumn{1}{c|}{0.586} & \multicolumn{1}{c|}{0.547} & \multicolumn{1}{c|}{0.603} & 0.561 & 0.574 $\pm$ 0.022               \\ \cline{2-8} 
                                       & \multicolumn{1}{c|}{}       & \chm   & \multicolumn{1}{c|}{0.687} & \multicolumn{1}{c|}{0.619} & \multicolumn{1}{c|}{0.664} & 0.645 & \textbf{0.654 $\pm$ 0.025}      \\ \hline\hline

    
    \multirow{2}{*}{SCIFACT,HEALTHVER} & \multicolumn{1}{c|}{\chm}   &        & \multicolumn{1}{c|}{0.580} & \multicolumn{1}{c|}{0.564} & \multicolumn{1}{c|}{0.551} & 0.570 & 0.566 $\pm$ 0.010               \\ \cline{2-8} 
                                       & \multicolumn{1}{c|}{}       & \chm   & \multicolumn{1}{c|}{0.664} & \multicolumn{1}{c|}{0.637} & \multicolumn{1}{c|}{0.657} & 0.649 & \textbf{0.652 $\pm$ 0.010}     \\ \hline
    
    \multirow{2}{*}{SCIFACT,SNLI}      & \multicolumn{1}{c|}{\chm}   &        & \multicolumn{1}{c|}{0.645} & \multicolumn{1}{c|}{0.607} & \multicolumn{1}{c|}{0.545} & 0.567 & 0.591 $\pm$ 0.039               \\ \cline{2-8} 
                                       & \multicolumn{1}{c|}{}       & \chm   & \multicolumn{1}{c|}{0.658} & \multicolumn{1}{c|}{0.574} & \multicolumn{1}{c|}{0.624} & 0.644 & 0.625 $\pm$ 0.032               \\ \hline
    
    \multirow{2}{*}{HEALTHVER,SNLI}    & \multicolumn{1}{c|}{\chm}   &        & \multicolumn{1}{c|}{0.615} & \multicolumn{1}{c|}{0.537} & \multicolumn{1}{c|}{0.615} & 0.658 & 0.606 $\pm$ 0.044               \\ \cline{2-8} 
                                       & \multicolumn{1}{c|}{}       & \chm   & \multicolumn{1}{c|}{0.667} & \multicolumn{1}{c|}{0.611} & \multicolumn{1}{c|}{0.664} & 0.602 & 0.636 $\pm$ 0.030               \\ \hline
    \end{tabular}%
    }
\end{table}

We have tested the two Dataset Expansion Strategies CombinedLoader (CombDL) and CombinedDataset (CombDS)
on two phases with the expasion Datasets SNLI, HEALTHVER and SCIFACT. The first Stage 
is combining one Dataset with SemEval Training and testing the performance only on SemEval
Validation. The second Stage including two datasets from the list and doing the same
Training and Validation as before. 


The Results (see Table~\ref{tab:my-dsexp}) showed that the CombinedLoader Strategy is worse than the Baseline, 
which indicates that the model has generalization issues if the dataset is present
with the same amount of Datapoints. Also, CombinedDataset Strategy is almost every
time better than the Baseline Model, with around 2\% Points gain of F1 Value, showing
that a generalized Model, handling multiple Text Entailment subtasks on different domain 
is possible. Secondly it is visible that SCIFACT in combination with SemEval has the highest
performance gain, which is the result of being tokenized around the same length of SemEval, therefore
the Model has to abstract more the general meaning of the sentences and the Dataset is not dominant,
meaning that the size is equally powerful.














\subsection{Sentence Embedding Architectures}

As Results (see Table~\ref{tab:my-seb}) of the different Architectures (defined in Chapter~\ref{ch:seb})
most of them are equally powerful than the Baseline. Version-3 is 1\% Point less than Version-4, which 
is expected due to we are giving more information to the Feed-Forward-Layer (FNN). Version-2 is better than
the Baseline but 4 times expensive in the backward calculations, which makes the performance boost neglectable.
Version-6, which is the Siamese-Network combined with Cosine Similarity of the two embeddings of the sentences
and MSELoss as Loss function, has a gain of 3.4\% F1-Value score, showing that Sentence Embedding can
help to improve the Metric-Scores on the Text Entailment Task.


\begin{table}[h]
    \centering
    \caption{F1 Values of the different Architecture from the Last (50th) Epoch, indicating that Sentence Embedding
             can boost the overall Performance.}
    \label{tab:my-seb}
    \begin{tabular}{|c|c||cccc|c|}
    \hline
    \multirow{2}{*}{Model} & \multirow{2}{*}{Desciption}                                                                       & \multicolumn{4}{c|}{Seed}                                                                    & \multirow{2}{*}{mean $\pm$ std} \\ \cline{3-6}
                           &                                                                                                   & \multicolumn{1}{c|}{0}     & \multicolumn{1}{c|}{42}    & \multicolumn{1}{c|}{1998}  & 1M    &                                 \\ \hline\hline
    Baseline               & \begin{tabular}[c]{@{}c@{}}BertModelForSequenceClassification\\ and CrossEntropyLoss\end{tabular} & \multicolumn{1}{c|}{0.636} & \multicolumn{1}{c|}{0.640} & \multicolumn{1}{c|}{0.638} & 0.657 & 0.630 $\pm$ 0.026               \\ \hline
    v2                     & \begin{tabular}[c]{@{}c@{}}(u, v, x, y) as input to FFN\\ and CrossEntropyLoss\end{tabular}       & \multicolumn{1}{c|}{0.652} & \multicolumn{1}{c|}{0.619} & \multicolumn{1}{c|}{0.604} & 0.664 & 0.635 $\pm$ 0.024               \\ \hline
    v3                     & \begin{tabular}[c]{@{}c@{}}(u, v) as input to FNN\\ and CrossEntropyLoss\end{tabular}             & \multicolumn{1}{c|}{0.708} & \multicolumn{1}{c|}{0.583} & \multicolumn{1}{c|}{0.573} & 0.611 & 0.619 $\pm$ 0.054               \\ \hline
    v4                     & \begin{tabular}[c]{@{}c@{}}(u, v, |u-v|) as input to FNN\\ and CrossEntropyLoss\end{tabular}      & \multicolumn{1}{c|}{0.638} & \multicolumn{1}{c|}{0.631} & \multicolumn{1}{c|}{0.618} & 0.640 & 0.632 $\pm$ 0.009               \\ \hline
    v5                     & \begin{tabular}[c]{@{}c@{}}CosSim(u, v)\\ and CosineEmbeddingLoss\end{tabular}                    & \multicolumn{1}{c|}{0.614} & \multicolumn{1}{c|}{0.603} & \multicolumn{1}{c|}{0.637} & 0.664 & 0.630 $\pm$ 0.023               \\ \hline
    v6                     & \begin{tabular}[c]{@{}c@{}}CosSim(u, v)\\ and MSELoss\end{tabular}                                & \multicolumn{1}{c|}{0.667} & \multicolumn{1}{c|}{0.675} & \multicolumn{1}{c|}{0.667} & 0.646 & \textbf{0.664 $\pm$ 0.011}               \\ \hline
    \end{tabular}
\end{table}











\newpage
\subsection{Adapter}

As said in Chapter~\ref{ch:adapter}, we are testing this Principle on SimCSE objective trained 
BERT Model and on general pre-trained BERT overridden the BertModelForSentenceClassification Class. Firstly the Adapter version of the general 
pre-trained BERT is around 2\% Points worse, which indicates that the remaining
trainable Parameters are not enough to get a meaningful representation for the Classification-Layer (see Table~\ref{tab:my-adapt}).
On the opposite side, taking the supervised fine-tuned SimCSE Model, which is trained on SNLI data, 
also a Text Entailment Task, shows a Performance boos from 1.5-2\% in F1 Score. Therefore, 
is important that the task correlate to each other, so the normal Layer of the BERT 
already have a good representation saved in the Tensor and the Adapter Layer only has to align these.


\begin{table}[!t]
    \centering
    \caption{F1 Values of the Adapter Trial from the Last (50th) Epoch for the general BERT Model and
             the supervise SimCSE objective trained BERT}
    \label{tab:my-adapt}
    \begin{tabular}{|l||cccc|c|}
    \hline
    \multicolumn{1}{|c||}{\multirow{2}{*}{Model}} & \multicolumn{4}{c|}{Seed}                                                                    & \multirow{2}{*}{mean $\pm$ std} \\ \cline{2-5}
    \multicolumn{1}{|c||}{}                       & \multicolumn{1}{c|}{0}     & \multicolumn{1}{c|}{42}    & \multicolumn{1}{c|}{1998}  & 1M    &                                 \\ \hline\hline
    Baseline BERT                                 & \multicolumn{1}{c|}{0.636} & \multicolumn{1}{c|}{0.640} & \multicolumn{1}{c|}{0.638} & 0.657 & 0.630 $\pm$ 0.026               \\ \hline
    Adapter BERT                                  & \multicolumn{1}{c|}{0.619} & \multicolumn{1}{c|}{0.587} & \multicolumn{1}{c|}{0.602} & 0.626 & 0.608 $\pm$ 0.015               \\ \hline
    Baseline SimCSE                               & \multicolumn{1}{c|}{0.598} & \multicolumn{1}{c|}{0.667} & \multicolumn{1}{c|}{0.696} & 0.640 & \textbf{0.650 $\pm$ 0.036}               \\ \hline
    Adapter SimCSE                                & \multicolumn{1}{c|}{0.643} & \multicolumn{1}{c|}{0.673} & \multicolumn{1}{c|}{0.615} & 0.654 & 0.646 $\pm$ 0.021      \\ \hline
    \end{tabular}
\end{table}

