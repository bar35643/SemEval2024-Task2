\section{Datasets}\label{sec:dataset}

\begin{table}[!b]
    \centering
    \caption{Two Example Clinical Trial Report taken out of the Dataset to represent the two types Comparison and Single.
             Each Sample has a Statement, Label, the relevant Section and one or two Premises called Trial \cite{noauthor_nli4ct_nodate-1}.}
    \label{tab:ds-example}
    \resizebox{\textwidth}{!}{%
    \begin{tabular}{|l||c|l|l|}
    \hline
    Type            & Short                                                         & Comparison                                                                                                                                                                                                                                                                                                                                                                            & Single                                                                                                                                                                                                                                                                                                                                                         \\ \hline\hline
    Label           &                                                               & Entailment                                                                                                                                                                                                                                                                                                                                                                            & Contradiction                                                                                                                                                                                                                                                                                                                                                  \\ \hline
    \textcolor{red}{Statement}       & \textcolor{red}{stm}                                                           & \textcolor{red}{\begin{tabular}[c]{@{}l@{}}The primary trial and the secondary \\ trial both used MRI for their interventions.\end{tabular}}                                                                                                                                                                                                                                                           & \textcolor{red}{\begin{tabular}[c]{@{}l@{}}More than 1/3 of patients in cohort 1 of the\\ primary trial experienced an adverse event.\end{tabular}}                                                                                                                                                                                                                             \\ \hline
    \textcolor{green}{Section}         & \textcolor{green}{sec}                                                           & \textcolor{green}{Intervention}                                                                                                                                                                                                                                                                                                                                                                          & \textcolor{green}{Adverse Events}                                                                                                                                                                                                                                                                                                                                                 \\ \hline
    \textcolor{blue}{Primary Trial}   & \textcolor{blue}{\begin{tabular}[c]{@{}l@{}}$p_1$\\ $p_2$\\ $...$\end{tabular}} & \textcolor{blue}{\begin{tabular}[c]{@{}l@{}}INTERVENTION 1:\\ •  Letrozole, Breast Enhancement, Safety\\ • Single arm of healthy postmenopausal women to \\    have two breast MRI (baseline and post-treatment). \\    Letrozole of 12.5 mg/day is given for three successive \\    days just prior to the second MRI.\\ • ...\end{tabular}}                                                           & \textcolor{blue}{\begin{tabular}[c]{@{}l@{}}Adverse Events 1:\\ •  Total: 69/258 (26.74\%)\\ •  Anaemia 3/258 (1.16\%)\\ •  Febrile neutropenia 13/258 (5.04\%)\\  •  Neutropenia 5/258 (1.94\%)\\  • ...\\ Adverse Events 2:\\ •  Mitral valve incompetence 0/224 (0.00\%)\\ •  Pericardial effusion 2/224 (0.89\%)\\ •  Sinus tachycardia 1/224 (0.45\%)\\ • ...\end{tabular}} \\ \hline
    \textcolor{blue}{Secondary Trial} & \textcolor{blue}{\begin{tabular}[c]{@{}l@{}}$s_1$\\ $s_2$\\ $...$\end{tabular}} & \textcolor{blue}{\begin{tabular}[c]{@{}l@{}}INTERVENTION 1: \\ •  Healthy Volunteers\\ • Healthy women will be screened for \\    Magnetic Resonance Imaging (MRI)  contraindications, \\    and then undergo contrast injection, and SWIFT acquisition.\\ •  Magnetic resonance imaging: Patients and healthy volunteers \\    will be first screened for MRI contraindications.\\ • ...\end{tabular}} &                                                                                                                                                                                                                                                                                                                                                                \\ \hline
    \end{tabular}%
    }
\end{table}

A group of four domain experts, including clinical trial organizers from the Manchester Cancer Institute and the 
Digital Experimental Cancer Medicine Team (DECMT), participated in an annotation task to generate entailment and contradiction 
statements for Clinical Trial Reports (CTR). Annotators were tasked with generating non-trivial statements about the contents of primary and 
secondary trials, encouraging understanding and reasoning. Each CTR was divided into four sections representing facts, and 
evidence supporting the labeled statement was selected from these facts. In cases of negation, the full CTR section was 
provided as evidence. A negative rewriting strategy was employed to create contradictory statements, and evidence contradicting 
these statements was collected \cite{jullien_semeval-2023_nodate} \cite{noauthor_nli4ct_nodate}.
Each Datapoint in the Training and Development Set have a statement, label being either Entangled or Contradiction, and one or two Prompts being
one of four different CTR sections of Adverse Events, Eligibility, Intervention or Results (see Table~\ref{tab:ds-example}). 
60\% of the Dataset are Single-Type, which then has only the Primary Prompt as relevant information and the other 40\% are 
Comparison-Type has Primary and Secondary Prompt \cite{noauthor_nli4ct_nodate}\cite{jullien_nli4ct_2023}\cite{jullien_semeval-2023_nodate}.
The SemEval Task 2 Dataset consists of 1700 Training and 200 Development samples (see Table~\ref{tab:ds-distribution}),
distributed equally between the two labels and almost for the 4 kinds of CTR sections. We use the development samples
as Validation Dataset to evaluate the performance of the Models.


\begin{table}[t]
    \centering
    \caption{Section and Label Distribution of the Training and Validation Dataset, showing
             an almost equal distribution through the four Sections \cite{noauthor_nli4ct_nodate-1}.}
    \label{tab:ds-distribution}
    \begin{tabular}{|c|c||cccc|c|c|}
    \hline
    \multirow{2}{*}{Dataset} & \multirow{2}{*}{Label} & \multicolumn{4}{c|}{Section} & \multirow{2}{*}{$\Sigma$} & \multirow{2}{*}{Total} \\
                           &               & Adverse Events & Eligibility & Intervention & Results &     &                       \\ \hline\hline
    \multirow{2}{*}{train} & Contradiction & 238            & 241         & 196          & 175     & 850 & \multirow{2}{*}{1700} \\
                           & Entailment    & 258            & 245         & 200          & 147     & 850 &                       \\ \hline
    \multirow{2}{*}{valid} & Contradiction & 26             & 28          & 18           & 28      & 100 & \multirow{2}{*}{200}  \\
                           & Entailment    & 26             & 28          & 18           & 28      & 100 &                       \\ \hline
    \end{tabular}
\end{table}


\begin{figure}[!b]
    \centering
    \includesvg[inkscapelatex=false, width=\textwidth]{./content/histogram}
    \caption{Histogram of Number of Words from the full Sentence and the tokenized version with BertTokenizer, 
             with 70\% are in line of 512 Tokens for both.}\label{cap:hist}
\end{figure}

For our further proceeding we concatenate the parts of a dataset sample as followed in Equation~\ref{eq:ds-sentence}.
The Single-Type, where there is no Second Trial (see Table~\ref{tab:ds-example}), $s_1, s_2, ... $
is empty, so the two last $[SEP]$ tokens are directly after another and for Comparison Type are all elements given, 
therefore the Sentence is exactly:

\begin{equation}\label{eq:ds-sentence}
\resizebox{0.91\hsize}{!}{%
 $Sentence = [CLS] \; \textcolor{red}{stm}\;
             [SEP] \; \textcolor{green}{sec}\;
             [SEP] \; \textcolor{blue}{p_1, p_2, ... }\;
             [SEP] \; \textcolor{blue}{s_1, s_2, ... }\;
             [SEP]$      
}
\end{equation}

As our base Model for our Experiments is BERT and the respective Tokenizer uses $512$ token, to encode the Sentence, 
we have analyzed the length Distribution in a Histogram (see Figure~\ref{cap:hist}). 
Round about 70\% of all Training and Validation Datapoints are in the range of the $512$ token length. 
Visible is also the slight shift on the $x$-Axis in the tokenized form, since not every word can have a single token
representing it.