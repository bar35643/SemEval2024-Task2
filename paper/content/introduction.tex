\section{Introduction}\label{sec:introduction}



In clinical studies, Clinical Trial Reports (CTRs) provide detailed information about patients
treatments and reactions \cite{noauthor_nli4ct_nodate} \cite{jullien_nli4ct_2023}. 
However, the increasing volume of CTRs, coupled with a lack of suitable 
analytical tools, poses challenges for clinicians to deliver personalized, evidence-based care \cite{takehana_stanford_2023}\cite{wang_knowcomp_2023}. 
Despite improvements in CTR reliability, there is a notable gap in the ability to analyze and compare them effectively \cite{vladika_sebis_2023}. 
Natural language inference (NLI) models show promise in deducing logical relationships in texts, but their application 
in the medical domain presents unique challenges \cite{wang_knowcomp_2023}\cite{alissa_just-km_2023}\cite{rajamanickam_i2r_2023}\cite{takehana_stanford_2023}\cite{vladika_sebis_2023}. 
These include the need for domain-specific knowledge due to clinical 
jargon and the variability in scientific communication. Additionally, distinguishing between non-entailed subsequences 
further complicates NLI systems \cite{noauthor_nli4ct_nodate}.


\paragraph{\textbf{Task:}} Classification of the relation between CTR premises and a 
statement, as being Entailed or a Contradiction. Models are expected to predict whether 
each statement affirms an entailment or forms a contradiction given the associated 
section from the claimed CTRs.


Recent works are reviewed, highlighting shortcomings in existing systems. 
Various models are examined, employing strategies such as pipeline concatenation, 
joint representation, supervised contrastive learning, and role-based enhancement \cite{wang_knowcomp_2023}\cite{alissa_just-km_2023}\cite{vladika_sebis_2023}\cite{zhou_thifly_2023}.
We are clarifying in Chapter~\ref{sec:dataset} how the CTR Dataset for the SemEval Task 2
is built and how we use them in our Models and Strategies. Also, we are briefly looking into the 
ranking of several, mostly BERT derivatives, Models.
Methods, detailed in the Chapter~\ref{sec:methods}, cover loss and metric functions, 
the use of BERT as the base model, sentence embedding architectures 
like SBERT, and adapter tuning. Experiments in constrained environments 
explore learning rate effects, token size challenges, and the impact of sentence 
permutation. Dataset expansion strategies and the effects of different sentence 
embedding architectures and adapter tuning on model performance are thoroughly evaluated.
The experiments (see Chapter~\ref{sec:experiments}) collectively shed light on the complexities of clinical trial report analysis, 
emphasizing the need to address challenges in data processing, model architecture, and training 
strategies.







