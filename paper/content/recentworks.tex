\section{Recent Works}\label{sec:recentworks}


\begin{table}[!tbh]
    \centering
    \caption{Results of the SemEval-2023 Task 7.1 (SemEval-2024 Task 2) \cite{noauthor_nli4ct_nodate}, with
             several attempts to use BERT as baseline.}
    \label{tab:results}
    \resizebox{\textwidth}{!}{%
    \begin{tabular}{|l||l|l|}
    \hline
    Model/Method     & Working Team                                                         & F1 Value                                                                                      \\ \hline\hline
    BERT             & \begin{tabular}[c]{@{}l@{}}\cite{wang_knowcomp_2023}KnowComp\\ \cite{vladika_sebis_2023}Sebis\\ \cite{alissa_just-km_2023}JUST-KM\end{tabular}   & \begin{tabular}[c]{@{}l@{}}base: 69.2 large: 70.9\\ base: 61:0\\ base: 63.4\end{tabular} \\ \hline
    DistilBERT       & \cite{takehana_stanford_2023}Standford                                                            & 60.8                                                                                     \\ \hline
    BioBERT          & \begin{tabular}[c]{@{}l@{}}\cite{vladika_sebis_2023}Sebis\\ \cite{takehana_stanford_2023}Standford\\ \cite{feng_ynu-hpcc_nodate}YNU-HPCC\end{tabular} & \begin{tabular}[c]{@{}l@{}}64.5\\ 63.7\\ 67.9\end{tabular}                               \\ \hline
    BioClinical-BERT & \begin{tabular}[c]{@{}l@{}}\cite{wang_knowcomp_2023}KnowComp\\ \cite{vladika_sebis_2023}Sebis\\ \cite{takehana_stanford_2023}Standford\end{tabular} & \begin{tabular}[c]{@{}l@{}}65.3\\ 65.7\\ 64.8\end{tabular}                               \\ \hline
    GatorTron-BERT   & \cite{alameldin_clemson_nodate}Clemson NLP                                                          & 70.5                                                                                     \\ \hline
    PubMedBERT       & \cite{takehana_stanford_2023}Standford                                                            & 66.0                                                                                     \\ \hline
    ALBERT-v2        & \cite{wang_knowcomp_2023}KnowComp                                                             & 67.1                                                                                     \\ \hline
    BART             & \cite{wang_knowcomp_2023}KnowComp                                                             & base: 67.1 large: 66.9                                                                   \\ \hline
    RoBERTa          & \begin{tabular}[c]{@{}l@{}}\cite{wang_knowcomp_2023}KnowComp\\ \cite{alissa_just-km_2023}JUST-KM\end{tabular}           & \begin{tabular}[c]{@{}l@{}}base: 70.7 large: 67.6\\ base: 65.6 large: 66.1 role-based: 67.0\end{tabular}  \\ \hline
    DeBERTa-v3       & \begin{tabular}[c]{@{}l@{}}\cite{wang_knowcomp_2023}KnowComp\\ \cite{vladika_sebis_2023}Sebis\end{tabular}             & \begin{tabular}[c]{@{}l@{}}base: 75.8 large: 81.5\\ large: 80.5\end{tabular}             \\ \hline
    ELECTRA          & \begin{tabular}[c]{@{}l@{}}\cite{wang_knowcomp_2023}KnowComp\\ \cite{takehana_stanford_2023}Standford\end{tabular}         & \begin{tabular}[c]{@{}l@{}}base:70.3 large: 76.1\\ small: 63.9\end{tabular}              \\ \hline
    GPT2             & \cite{wang_knowcomp_2023}KnowComp                                                             & base: 39.0 medium 44.2 large: 61.5                                                       \\ \hline
    T5               & \cite{rajamanickam_i2r_2023}I2R                                                                  & base: 62.9 large: 68.3                                                                   \\ \hline
    Flan-T5-xxl      & \cite{kanakarajan_saama_2023}Saama                                                                & 83.4                                                                                     \\ \hline
    MGNet            & \cite{zhou_thifly_2023}THiFLY                                                               & 85.6                                                                                     \\ \hline
    \end{tabular}%
    }
\end{table}


The majority of the released systems failed to achieve significantly above the majority-class baseline of 66.7\%
F1 value (see Table~\ref{tab:results}) \cite{jullien_semeval-2023_nodate}.
\textbf{Sebis \cite{vladika_sebis_2023}} uses a system concatenating the parts of the CTRs in two different method, called pipeline and joint, 
where basically the sentence representation includes more $[SEP]$ tokens to dense it up.
\textbf{KnowComp \cite{wang_knowcomp_2023}, YNU-HPCC \cite{feng_ynu-hpcc_nodate}, Stanford \cite{takehana_stanford_2023}, I2R \cite{rajamanickam_i2r_2023} and Clemson NLP \cite{alameldin_clemson_nodate}} uses the same strategy as we do (see Equation ~\ref{eq:ds-sentence}) to feed forward the sentence in several most popular Models.
Compared to the others, \textbf{YNU-HPCC \cite{feng_ynu-hpcc_nodate}} utilize Supervised Contrastive Learning with the corresponding loss function, 
to maximize, if it is contradiction, or minimize, if it is entailed, the spatial representation vector of the two compared inputs.
\textbf{JUST-KM \cite{alissa_just-km_2023}} Models are enhancing RoBERTa in a role-based approach, where the two RoBERTa-Large Models trained differently, 
to predict the general outcome.
\textbf{Saama \cite{kanakarajan_saama_2023}} finetuned Flan-T5 LLM to SemEval's task-specific data by applying different Instruction Templates. 
\textbf{THiFLY \cite{zhou_thifly_2023}} employ Multigranularity Inference Network, which uses the Equation~\ref{eq:ds-sentence} sentence structure to pass
it further to a Joint Semantic Encoder followed by Pooling and Sencence-Level Encoder before Classification.


