\documentclass[runningheads]{llncs}
%
\usepackage[T1]{fontenc}
% T1 fonts will be used to generate the final print and online PDFs,
% so please use T1 fonts in your manuscript whenever possible.
% Other font encondings may result in incorrect characters.
%
\usepackage{amsmath}
\usepackage{graphicx}
\usepackage{nameref}
\usepackage{varioref}
\usepackage{hyperref}
\usepackage{cleveref}
\usepackage{multirow}
\usepackage{svg}

\usepackage{longtable}
\usepackage{hhline}
% Used for displaying a sample figure. If possible, figure files should
% be included in EPS format.
%
% If you use the hyperref package, please uncomment the following two lines
% to display URLs in blue roman font according to Springer's eBook style:
\usepackage{color, colortbl}
%\renewcommand\UrlFont{\color{blue}\rmfamily}
%
\usepackage{tikz}
\usepackage{tikzit}
\input{sample.tikzstyles}
\def\chm{\tikz\fill[scale=0.4](0,.35) -- (.25,0) -- (1,.7) -- (.25,.15) -- cycle;}
\def\crs{\tikz[scale=0.25, line width=0.3mm]\draw(0,0) -- (1,1) (0,1) -- (1,0);}

\def\pbox{\parbox{6cm}}

\begin{document}
\title{
    SemEval 2024 Task 2: Safe Biomedical Natural Language Inference for Clinical Trials
    %\thanks{Supported by organization x.}
}
%
%\titlerunning{Abbreviated paper title}
% If the paper title is too long for the running head, you can set
% an abbreviated paper title here
%
\author{Raphael Baumann\inst{1}\orcidID{2909063}}
%
\authorrunning{R. Baumann}
% First names are abbreviated in the running head.
% If there are more than two authors, 'et al.' is used.
%
\institute{
Chair of Natural Language Processing \\ 
Institute of Computer Science \\
Faculty of Mathematics and Computer Science \\ 
Julius-Maximilians University of Würzburg \\ 
Würzburg, Germany\\
\email{raphael.baumann@stud-mail.uni-wuerzburg.de} \\
}




\maketitle              % typeset the header of the contribution

\begingroup %\input{} or %\include{}
\begin{abstract}
	This paper focuses on the creation and utilization of a specialized dataset 
	for enhancing natural language processing models' comprehension of 
	Clinical Trial Reports (CTR). The resulting dataset, structured as 
	Single-Type and Comparison-Type, serves as the foundation for 
	training and development sets.
	We explored methodologies, utilizing BERT as the base model, 
	and delves into strategies like Sentence Embedding Architectures, 
	Adapter Tuning, and different Loss Functions. It systematically evaluates 
	factors such as learning rates, sentence permutations, 
	and dataset expansion strategies, to generalize a Model, handling,
	like a translation task, several Test Entailment Task at once.


	\keywords{Natural Language Processing (NLP)\and
		Large Language Models (LLMs)  \and
		medical \and
		BERT \and
		Transformer \and
		Machine Learning \and
		Artificial  Intelligence}
\end{abstract}
%\newpage
\section{Introduction}\label{sec:introduction}

\textcolor{red}{TODO: Results to last Epoch instead of Best}

\textcolor{red}{TODO: Add Cotations}


\paragraph{\textbf{Task:}} Classification of the relation between CTR premises and a 
statement, as being Entailed or a Contradiction. Models are expected to predict whether 
each statement affirms an entailment or forms a contradiction given the associated 
section from the claimed CTRs.








\newpage
\section{Datasets}\label{sec:dataset}

A group of four domain experts, including clinical trial organizers from the Manchester Cancer Institute and the 
Digital Experimental Cancer Medicine Team (DECMT), participated in an annotation task to generate entailment and contradiction 
statements for Clinical Trial Reports (CTR). Annotators were tasked with generating non-trivial statements about the contents of primary and 
secondary trials, encouraging understanding and reasoning. Each CTR was divided into four sections representing facts, and 
evidence supporting the labeled statement was selected from these facts. In cases of negation, the full CTR section was 
provided as evidence. A negative rewriting strategy was employed to create contradictory statements, and evidence contradicting 
these statements was collected \cite{jullien_semeval-2023_nodate} \cite{noauthor_nli4ct_nodate}.
Each Datapoint in the Training and Development Set have a statement, label being either Entangled or Contradiction, and one or two Prompts being
one of four different CTR sections of Adverse Events, Eligibility, Intervention or Results (see Table~\ref{tab:ds-example}). 
60\% of the Dataset are Single-Type, which then has only the Primary Prompt as relevant information and the other 40\% are 
Comparison-Type has Primary and Secondary Prompt \cite{noauthor_nli4ct_nodate}\cite{jullien_nli4ct_2023}\cite{jullien_semeval-2023_nodate}.
The SemEval task 2 dataset consists of 1700 training and 200 development samples (see Table~\ref{tab:ds-distribution}),
distributed equally between the two labels and almost for the 4 kinds of CTR sections. We use the development samples
as validation dataset to evaluate the performance of the Models.

\begin{table}[!h]
    \centering
    \caption{Two Example Clinical Trial Report taken out of the Dataset to represent the two types Comparison and Single.
             Each Sample has a Statement, Label, the relevant Section and one or two Premises called Trial.}
    \label{tab:ds-example}
    \resizebox{\textwidth}{!}{%
    \begin{tabular}{|l||c|l|l|}
    \hline
    Type            & Short                                                         & Comparison                                                                                                                                                                                                                                                                                                                                                                            & Single                                                                                                                                                                                                                                                                                                                                                         \\ \hline\hline
    Label           &                                                               & Entailment                                                                                                                                                                                                                                                                                                                                                                            & Contradiction                                                                                                                                                                                                                                                                                                                                                  \\ \hline
    \textcolor{red}{Statement}       & \textcolor{red}{stm}                                                           & \textcolor{red}{\begin{tabular}[c]{@{}l@{}}The primary trial and the secondary \\ trial both used MRI for their interventions.\end{tabular}}                                                                                                                                                                                                                                                           & \textcolor{red}{\begin{tabular}[c]{@{}l@{}}More than 1/3 of patients in cohort 1 of the\\ primary trial experienced an adverse event.\end{tabular}}                                                                                                                                                                                                                             \\ \hline
    \textcolor{green}{Section}         & \textcolor{green}{sec}                                                           & \textcolor{green}{Intervention}                                                                                                                                                                                                                                                                                                                                                                          & \textcolor{green}{Adverse Events}                                                                                                                                                                                                                                                                                                                                                 \\ \hline
    \textcolor{blue}{Primary Trial}   & \textcolor{blue}{\begin{tabular}[c]{@{}l@{}}$p_1$\\ $p_2$\\ $...$\end{tabular}} & \textcolor{blue}{\begin{tabular}[c]{@{}l@{}}INTERVENTION 1:\\ •  Letrozole, Breast Enhancement, Safety\\ • Single arm of healthy postmenopausal women to \\    have two breast MRI (baseline and post-treatment). \\    Letrozole of 12.5 mg/day is given for three successive \\    days just prior to the second MRI.\\ • ...\end{tabular}}                                                           & \textcolor{blue}{\begin{tabular}[c]{@{}l@{}}Adverse Events 1:\\ •  Total: 69/258 (26.74\%)\\ •  Anaemia 3/258 (1.16\%)\\ •  Febrile neutropenia 13/258 (5.04\%)\\  •  Neutropenia 5/258 (1.94\%)\\  • ...\\ Adverse Events 2:\\ •  Mitral valve incompetence 0/224 (0.00\%)\\ •  Pericardial effusion 2/224 (0.89\%)\\ •  Sinus tachycardia 1/224 (0.45\%)\\ • ...\end{tabular}} \\ \hline
    \textcolor{blue}{Secondary Trial} & \textcolor{blue}{\begin{tabular}[c]{@{}l@{}}$s_1$\\ $s_2$\\ $...$\end{tabular}} & \textcolor{blue}{\begin{tabular}[c]{@{}l@{}}INTERVENTION 1: \\ •  Healthy Volunteers\\ • Healthy women will be screened for \\    Magnetic Resonance Imaging (MRI)  contraindications, \\    and then undergo contrast injection, and SWIFT acquisition.\\ •  Magnetic resonance imaging: Patients and healthy volunteers \\    will be first screened for MRI contraindications.\\ • ...\end{tabular}} &                                                                                                                                                                                                                                                                                                                                                                \\ \hline
    \end{tabular}%
    }
\end{table}


\begin{table}[!h]
    \centering
    \caption{Section and Label Distribution of the Training and Validation Dataset, showing
             an almost equal distribution torough the four Sections.}
    \label{tab:ds-distribution}
    \begin{tabular}{|c|c||cccc|c|c|}
    \hline
    \multirow{2}{*}{Dataset} & \multirow{2}{*}{Label} & \multicolumn{4}{c|}{Section} & \multirow{2}{*}{$\Sigma$} & \multirow{2}{*}{Total} \\
                           &               & Adverse Events & Eligibility & Intervention & Results &     &                       \\ \hline\hline
    \multirow{2}{*}{train} & Contradiction & 238            & 241         & 196          & 175     & 850 & \multirow{2}{*}{1700} \\
                           & Entailment    & 258            & 245         & 200          & 147     & 850 &                       \\ \hline
    \multirow{2}{*}{valid} & Contradiction & 26             & 28          & 18           & 28      & 100 & \multirow{2}{*}{200}  \\
                           & Entailment    & 26             & 28          & 18           & 28      & 100 &                       \\ \hline
    \end{tabular}
\end{table}

\newpage


For our further proceeding we concatenate the parts of a dataset sample as followed in Equation~\ref{eq:ds-sentence}.
The Single-Type, where there is no Second Trial (see Table~\ref{tab:ds-example}), $s_1, s_2, ... $
is empty, so the two last $[SEP]$ tokens are directly after another and for Comparison Type are all elements given, 
therefore the Sentence is exactly:

\begin{equation}\label{eq:ds-sentence}
\resizebox{0.91\hsize}{!}{%
 $Sentence = [CLS] \; \textcolor{red}{stm}\;
             [SEP] \; \textcolor{green}{sec}\;
             [SEP] \; \textcolor{blue}{p_1, p_2, ... }\;
             [SEP] \; \textcolor{blue}{s_1, s_2, ... }\;
             [SEP]$      
}
\end{equation}

As our base Model for our Experiments is BERT and the respective Tokenizer uses $512$ token, to encode the Sentence, 
we have analyzed the length Distribution in a Histogram (see Figure~\ref{cap:hist}). 
Round about 70\% of all Training and Validation Datapoints are in the range of the $512$ token length. 
Visible is also the slight shift on the $x$-Axis in the tokenized form, since not every word can have a single token
representing it.

\begin{figure}[!htb]
    \hspace*{-4.5cm}
    \centering
    \includesvg[inkscapelatex=false, width=1.65\textwidth]{./content/histogram}
    \caption{Histogram of Number of Words from the full Sentence and the tokenized version with BertTokenizer, 
             whith 70\% are in line of 512 Tokens for both.}\label{cap:hist}
\end{figure}
\section{Recent Works}\label{sec:recentworks}


\begin{table}[!h]
    \centering
    \caption{Results of the SemEval-2023 Task 7.1 (SemEval-2024 Task 2) \cite{noauthor_nli4ct_nodate}, with
             several attempts to use BERT as baseline.}
    \label{tab:results}
    \resizebox{\textwidth}{!}{%
    \begin{tabular}{|l||l|l|}
    \hline
    Model/Method     & Working Team                                                         & F1 Value                                                                                      \\ \hline\hline
    BERT             & \begin{tabular}[c]{@{}l@{}}\cite{wang_knowcomp_2023}KnowComp\\ \cite{vladika_sebis_2023}Sebis\\ \cite{alissa_just-km_2023}JUST-KM\end{tabular}   & \begin{tabular}[c]{@{}l@{}}base: 69.2 large: 70.9\\ base: 61:0\\ base: 63.4\end{tabular} \\ \hline
    DistilBERT       & \cite{takehana_stanford_2023}Standford                                                            & 60.8                                                                                     \\ \hline
    BioBERT          & \begin{tabular}[c]{@{}l@{}}\cite{vladika_sebis_2023}Sebis\\ \cite{takehana_stanford_2023}Standford\\ \cite{feng_ynu-hpcc_nodate}YNU-HPCC\end{tabular} & \begin{tabular}[c]{@{}l@{}}64.5\\ 63.7\\ 67.9\end{tabular}                               \\ \hline
    BioClinical-BERT & \begin{tabular}[c]{@{}l@{}}\cite{wang_knowcomp_2023}KnowComp\\ \cite{vladika_sebis_2023}Sebis\\ \cite{takehana_stanford_2023}Standford\end{tabular} & \begin{tabular}[c]{@{}l@{}}65.3\\ 65.7\\ 64.8\end{tabular}                               \\ \hline
    GatorTron-BERT   & \cite{alameldin_clemson_nodate}Clemson NLP                                                          & 70.5                                                                                     \\ \hline
    PubMedBERT       & \cite{takehana_stanford_2023}Standford                                                            & 66.0                                                                                     \\ \hline
    ALBERT-v2        & \cite{wang_knowcomp_2023}KnowComp                                                             & 67.1                                                                                     \\ \hline
    BART             & \cite{wang_knowcomp_2023}KnowComp                                                             & base: 67.1 large: 66.9                                                                   \\ \hline
    RoBERTa          & \begin{tabular}[c]{@{}l@{}}\cite{wang_knowcomp_2023}KnowComp\\ \cite{alissa_just-km_2023}JUST-KM\end{tabular}           & \begin{tabular}[c]{@{}l@{}}base: 70.7 large: 67.6\\ base: 65.6 large: 66.1 role-based: 67.0\end{tabular}  \\ \hline
    DeBERTa-v3       & \begin{tabular}[c]{@{}l@{}}\cite{wang_knowcomp_2023}KnowComp\\ \cite{vladika_sebis_2023}Sebis\end{tabular}             & \begin{tabular}[c]{@{}l@{}}base: 75.8 large: 81.5\\ large: 80.5\end{tabular}             \\ \hline
    ELECTRA          & \begin{tabular}[c]{@{}l@{}}\cite{wang_knowcomp_2023}KnowComp\\ \cite{takehana_stanford_2023}Standford\end{tabular}         & \begin{tabular}[c]{@{}l@{}}base:70.3 large: 76.1\\ small: 63.9\end{tabular}              \\ \hline
    GPT2             & \cite{wang_knowcomp_2023}KnowComp                                                             & base: 39.0 medium 44.2 large: 61.5                                                       \\ \hline
    T5               & \cite{rajamanickam_i2r_2023}I2R                                                                  & base: 62.9 large: 68.3                                                                   \\ \hline
    Flan-T5-xxl      & \cite{kanakarajan_saama_2023}Saama                                                                & 83.4                                                                                     \\ \hline
    MGNet            & \cite{zhou_thifly_2023}THiFLY                                                               & 85.6                                                                                     \\ \hline
    \end{tabular}%
    }
\end{table}


The majority of the released systems failed to achieve significantly above the majority-class baseline of 66.7\%
F1 value (see Table~\ref{tab:results}) \cite{jullien_semeval-2023_nodate}.
\textbf{Sebis \cite{vladika_sebis_2023}} uses a system concatenating the parts of the CTRs in two different method, called pipeline and joint, 
where basically the sentence representation includes more $[SEP]$ tokens to dense it up.
\textbf{KnowComp \cite{wang_knowcomp_2023}, YNU-HPCC \cite{feng_ynu-hpcc_nodate}, Stanford \cite{takehana_stanford_2023}, I2R \cite{rajamanickam_i2r_2023} and Clemson NLP \cite{alameldin_clemson_nodate}} uses the same strategy as we do (see Equation ~\ref{eq:ds-sentence}) to feed forward the sentence in several most popular Models.
Compared to the others, \textbf{YNU-HPCC \cite{feng_ynu-hpcc_nodate}} utilize Supervised Contrastive Learning with the corresponding loss function, 
to maximize, if it is contradiction, or minimize, if it is entailed, the spatial representation vector of the two compared inputs.
\textbf{JUST-KM \cite{alissa_just-km_2023}} Models are enhancing RoBERTa in a role-based approach, where the two RoBERTa-Large Models trained differently, 
to predict the general outcome.
\textbf{Saama \cite{kanakarajan_saama_2023}} finetuned Flan-T5 LLM to SemEval's task-specific data by applying different Instruction Templates. 
\textbf{THiFLY \cite{zhou_thifly_2023}} employ Multigranularity Inference Network, which uses the Equation~\ref{eq:ds-sentence} sentence structure to pass
it further to a Joint Semantic Encoder followed by Pooling and Sencence-Level Encoder before Classification.



\section{Methods}\label{sec:methods}
In this chapter we are clarifying the Loss and Metric Functions, the Architecture BERT used for our Experiments, Ideas principles like Adapter Tuning and
Sentence Embedding and how fusing different Dataset together works, which are used in training loop.


\subsection{BERT}
\begin{figure}[h]
    \centering
    \includegraphics[scale=0.33]{./content/BERT_Architecture.png}
    \caption{Bidirectional Encoder Representations from Transformers (BERT) 
             scematic View \cite{devlin_bert_2019}.}
    \label{tab:bert}
\end{figure}

The Experiments are using BERT, Bidirectional Encoder Representations from Transformers (see Fig.~\ref{tab:bert}) ~\cite{devlin_bert_2019} 
developed by Google is based on Transformer architecture introduced by Vaswani et. al. \cite{vaswani_attention_2023}.
Unlike previous attempts, that process text in a unidirectional way (either left to right or right to 
left), BERT is designed to understand context bidirectionally as every Token is connected Pathways with every other. A masked language model (MLM) 
pre-training target is used, where tokens are randomly masked from the input to predict the original 
vocabulary IDs.
The model can be fine-tuned for specific downstream tasks, such as classification or translation. 
BERT is available in different sizes like BERT-Base and BERT-Large. There are various implementations 
such as RoBERTa~\cite{liu_roberta_2019}, ALBERT~\cite{lan_albert_2020}, BART~\cite{lewis_bart_2020}, 
DeBERTa~\cite{he_deberta_2021}, which improves BERT architecture differently.














\subsection{Sentence Embedding Architectures}
\textcolor{red}{TODO}

















\subsection{Adapter Tuning}
\begin{figure}[h]
    \centering
    \includegraphics[scale=0.2]{./content/Adapter_Architecture.png}
    \caption{Basic Structure of Adapter built on top of a Models Architecture, where only the Adapter Layers Parameters are trainable \cite{zheng_learn_2023}.}
    \label{tab:bert}
\end{figure}

Adapter Tuning is a supervised method, where input, gold label are given and the models parameters are frozen, 
but adding new fully trainable bottleneck feed-forward networks 
on each intermediate layer. The objective is to reduce the size of trainable parameters, 
to gain higher throughput and keeping the pre-trained embeddings\cite{zheng_learn_2023} \cite{naveed_comprehensive_2023}. 
The ultimate goal of adaptation training is to enhance the model's scores on the downstream task, 
while still benefiting from the broad language understanding gained during the initial 
pre-training \cite{manjavacas_adapting_2022}. The effectiveness of adaptation-tuning depends on 
the similarity between the pre-training task and the target task due to fixed embeddings.













\subsection{Fusing different Datasets/Loaders}
\begin{figure}[h]
    \centering
    \resizebox{\textwidth}{!}{\begin{tikzpicture}
	\begin{pgfonlayer}{nodelayer}
		\node [style=base] (0) at (0, 0) {};
		\node [style=1x1 v1] (1) at (-7.5, 0) {};
		\node [style=1x1 v1] (4) at (-6.5, 0) {};
		\node [style=1x1 v1] (6) at (-5.5, 0) {};
		\node [style=1x1 v1] (7) at (-4.5, 0) {};
		\node [style=1x1 v3] (9) at (-3.5, 0) {};
		\node [style=1x1 v3] (10) at (-2.5, 0) {};
		\node [style=1x1 v3] (11) at (-1.5, 0) {};
		\node [style=1x1 v3] (12) at (-0.5, 0) {};
		\node [style=1x1 v4] (35) at (0.5, 0) {};
		\node [style=1x1 v4] (36) at (1.5, 0) {};
		\node [style=1x1 v4] (37) at (2.5, 0) {};
		\node [style=1x1 v4] (38) at (3.5, 0) {};
		\node [style=1x1 v2] (39) at (4.5, 0) {};
		\node [style=1x1 v2] (40) at (5.5, 0) {};
		\node [style=1x1 v2] (41) at (6.5, 0) {};
		\node [style=1x1 v2] (42) at (7.5, 0) {};
		\node [style=base] (43) at (19, 0) {};
		\node [style=1x1 v1] (44) at (11.5, 0) {};
		\node [style=1x1 v1] (45) at (12.5, 0) {};
		\node [style=1x1 v1] (46) at (13.5, 0) {};
		\node [style=1x1 v1] (47) at (14.5, 0) {};
		\node [style=1x1 v3] (48) at (15.5, 0) {};
		\node [style=1x1 v3] (49) at (16.5, 0) {};
		\node [style=1x1 v4] (52) at (17.5, 0) {};
		\node [style=1x1 v2] (56) at (19.5, 0) {};
		\node [style=1x1 v2] (57) at (24.5, 0) {};
		\node [style=1x1 v2] (58) at (25.5, 0) {};
		\node [style=1x1 v2] (59) at (26.5, 0) {};
		\node [style=base] (60) at (0, -3) {};
		\node [style=1x1 v1] (61) at (-7.5, -3) {};
		\node [style=1x1 v1] (62) at (-6.5, -3) {};
		\node [style=1x1 v1] (63) at (-5.5, -3) {};
		\node [style=1x1 v1] (64) at (-4.5, -3) {};
		\node [style=1x1 v3] (65) at (-3.5, -3) {};
		\node [style=1x1 v3] (66) at (-2.5, -3) {};
		\node [style=1x1 v3] (67) at (-1.5, -3) {};
		\node [style=1x1 v3] (68) at (-0.5, -3) {};
		\node [style=1x1 v4] (69) at (0.5, -3) {};
		\node [style=1x1 v4] (70) at (1.5, -3) {};
		\node [style=1x1 v4] (71) at (2.5, -3) {};
		\node [style=1x1 v4] (72) at (3.5, -3) {};
		\node [style=1x1 v2] (73) at (4.5, -3) {};
		\node [style=1x1 v2] (74) at (5.5, -3) {};
		\node [style=1x1 v2] (75) at (6.5, -3) {};
		\node [style=1x1 v2] (76) at (7.5, -3) {};
		\node [style=base] (77) at (0, -1.5) {};
		\node [style=1x1 v1] (78) at (-7.5, -1.5) {};
		\node [style=1x1 v1] (79) at (-6.5, -1.5) {};
		\node [style=1x1 v1] (80) at (-5.5, -1.5) {};
		\node [style=1x1 v1] (81) at (-4.5, -1.5) {};
		\node [style=1x1 v3] (82) at (-3.5, -1.5) {};
		\node [style=1x1 v3] (83) at (-2.5, -1.5) {};
		\node [style=1x1 v3] (84) at (-1.5, -1.5) {};
		\node [style=1x1 v3] (85) at (-0.5, -1.5) {};
		\node [style=1x1 v4] (86) at (0.5, -1.5) {};
		\node [style=1x1 v4] (87) at (1.5, -1.5) {};
		\node [style=1x1 v4] (88) at (2.5, -1.5) {};
		\node [style=1x1 v4] (89) at (3.5, -1.5) {};
		\node [style=1x1 v2] (90) at (4.5, -1.5) {};
		\node [style=1x1 v2] (91) at (5.5, -1.5) {};
		\node [style=1x1 v2] (92) at (6.5, -1.5) {};
		\node [style=1x1 v2] (93) at (7.5, -1.5) {};
		\node [style=base] (94) at (0, -4.5) {};
		\node [style=1x1 v1] (95) at (-7.5, -4.5) {};
		\node [style=1x1 v1] (96) at (-6.5, -4.5) {};
		\node [style=1x1 v1] (97) at (-5.5, -4.5) {};
		\node [style=1x1 v1] (98) at (-4.5, -4.5) {};
		\node [style=1x1 v3] (99) at (-3.5, -4.5) {};
		\node [style=1x1 v3] (100) at (-2.5, -4.5) {};
		\node [style=1x1 v3] (101) at (-1.5, -4.5) {};
		\node [style=1x1 v3] (102) at (-0.5, -4.5) {};
		\node [style=1x1 v4] (103) at (0.5, -4.5) {};
		\node [style=1x1 v4] (104) at (1.5, -4.5) {};
		\node [style=1x1 v4] (105) at (2.5, -4.5) {};
		\node [style=1x1 v4] (106) at (3.5, -4.5) {};
		\node [style=1x1 v2] (107) at (4.5, -4.5) {};
		\node [style=1x1 v2] (108) at (5.5, -4.5) {};
		\node [style=1x1 v2] (109) at (6.5, -4.5) {};
		\node [style=1x1 v2] (110) at (7.5, -4.5) {};
		\node [style=1x1 v1] (111) at (20.5, 0) {};
		\node [style=1x1 v1] (112) at (21.5, 0) {};
		\node [style=1x1 v3] (113) at (22.5, 0) {};
		\node [style=1x1 v2] (114) at (18.5, 0) {};
		\node [style=1x1 v4] (127) at (23.5, 0) {};
		\node [style=base] (128) at (19, -1.5) {};
		\node [style=1x1 v1] (129) at (11.5, -1.5) {};
		\node [style=1x1 v1] (130) at (12.5, -1.5) {};
		\node [style=1x1 v1] (131) at (13.5, -1.5) {};
		\node [style=1x1 v1] (132) at (14.5, -1.5) {};
		\node [style=1x1 v1] (140) at (20.5, -1.5) {};
		\node [style=1x1 v1] (141) at (21.5, -1.5) {};
		\node [style=1x1 v4] (144) at (19.5, -1.5) {};
		\node [style=1x1 v1] (145) at (15.5, -1.5) {};
		\node [style=1x1 v1] (146) at (16.5, -1.5) {};
		\node [style=1x1 v1] (147) at (17.5, -1.5) {};
		\node [style=1x1 v1] (148) at (18.5, -1.5) {};
		\node [style=1x1 v1] (149) at (22.5, -1.5) {};
		\node [style=1x1 v1] (150) at (23.5, -1.5) {};
		\node [style=1x1 v1] (151) at (24.5, -1.5) {};
		\node [style=1x1 v1] (152) at (25.5, -1.5) {};
		\node [style=1x1 v1] (153) at (26.5, -1.5) {};
		\node [style=base] (154) at (19, -3) {};
		\node [style=1x1 v1] (155) at (15.5, -3) {};
		\node [style=1x1 v1] (159) at (20.5, -3) {};
		\node [style=1x1 v1] (160) at (21.5, -3) {};
		\node [style=1x1 v4] (161) at (19.5, -3) {};
		\node [style=1x1 v1] (165) at (18.5, -3) {};
		\node [style=1x1 v1] (166) at (22.5, -3) {};
		\node [style=1x1 v1] (167) at (23.5, -3) {};
		\node [style=1x1 v1] (168) at (24.5, -3) {};
		\node [style=1x1 v1] (170) at (26.5, -3) {};
		\node [style=1x1 v4] (171) at (11.5, -3) {};
		\node [style=1x1 v4] (172) at (25.5, -3) {};
		\node [style=1x1 v4] (173) at (14.5, -3) {};
		\node [style=1x1 v3] (174) at (12.5, -3) {};
		\node [style=1x1 v2] (175) at (13.5, -3) {};
		\node [style=1x1 v2] (176) at (16.5, -3) {};
		\node [style=1x1 v2] (177) at (17.5, -3) {};
		\node [style=base] (178) at (19, -4.5) {};
		\node [style=1x1 v4] (182) at (19.5, -4.5) {};
		\node [style=1x1 v1] (183) at (18.5, -4.5) {};
		\node [style=1x1 v1] (185) at (23.5, -4.5) {};
		\node [style=1x1 v1] (186) at (24.5, -4.5) {};
		\node [style=1x1 v1] (187) at (26.5, -4.5) {};
		\node [style=1x1 v4] (188) at (12.5, -4.5) {};
		\node [style=1x1 v4] (189) at (25.5, -4.5) {};
		\node [style=1x1 v3] (191) at (11.5, -4.5) {};
		\node [style=1x1 v2] (192) at (13.5, -4.5) {};
		\node [style=1x1 v2] (193) at (16.5, -4.5) {};
		\node [style=1x1 v2] (194) at (17.5, -4.5) {};
		\node [style=1x1 v2] (195) at (14.5, -4.5) {};
		\node [style=1x1 v2] (196) at (15.5, -4.5) {};
		\node [style=1x1 v2] (197) at (20.5, -4.5) {};
		\node [style=1x1 v2] (198) at (21.5, -4.5) {};
		\node [style=1x1 v3] (199) at (22.5, -4.5) {};
		\node [style=Textfeld] (200) at (19, 1.25) {Combined Dataset};
		\node [style=Textfeld] (201) at (0, 1.25) {Combined Loader};
		\node [style=Textfeld] (202) at (9.5, 1.25) {Batch};
		\node [style=Textfeld] (203) at (9.5, 0) {1};
		\node [style=Textfeld] (204) at (9.5, -1.5) {2};
		\node [style=Textfeld] (205) at (9.5, -3) {...};
		\node [style=Textfeld] (206) at (9.5, -4.5) {n};
	\end{pgfonlayer}
\end{tikzpicture}
}
    \caption{Combined Dataset and Combined Loader Strategy where 4 different Dataset are getting mixed together.
             On the Loader-Level for a Batch size of 16, every Dataset is present equally and on the
             Dataset-Level, they are concatenated and the mixed.}\label{fig:test}
\end{figure}
Normally a Pre-Trained Model is used which is then fine-tuned on the Specific Dataset. The same can be applied to
Text Classification. Therefore the Model is performing only on SemEval very good, while mismatching on generalization
on other Text Classification Tasks. To Compensate, two strategies are applied to solve this.
CombinedLoaders is a strategy, where each Dataset, on which the Model should perform, can contribute equally with
the same amount of Datapoints. Another strategy involves the concatenation of all these Datasets to one and then being
mixed on the Training. For our Example (see Chapter~\ref{ch:ds}) we are using SNLI \cite{noauthor_snli_2023}
which has short sencences for entailment check, Healthver \cite{noauthor_dwaddenhealthver_entailment_nodate} a Dataset on the medical domain and
scifact \cite{noauthor_allenaiscifact_entailment_nodate} on the scientific domain, to check the entailment of claim, 
paper title and abstract. The Evaluation of the Strategies is only applied to SemEval Task Development
Dataset not on the choosen ones for expanding. Also the expected outcome should be lower than the
specialized Model on SemEval as it has to generalize the different domains more.























\subsection{Loss Functions and Evaluation Metric}
We are using 3 commonly used Loss functions for our Training of the different
Architectures with $x$ being the Prediction and $y$ being the searched Class. For direct Classification, 
CrossEntropyLoss~\ref{eq:ce} is the commonly and most frequently used function. Secondly, as we see later,
combined with Cosine Simmilarity, we are using MSELoss~\ref{eq:mse} or CosineEmbeddingLoss~\ref{eq:ceb}
to maximize or minimize the distance between two representations. As Metric the SemEval Task uses F1 Score, which
is the harmonic mean between precision and recall defined in Equation~\ref{eq:f} \cite{noauthor_nli4ct_nodate}.



\begin{equation}\label{eq:ce}
   Loss_{CE} = -\sum_{i=1}^My_{o,i}\log(x_{o,i})
\end{equation}
\begin{equation}\label{eq:mse}
   Loss_{MSE} = \sum_{i=1}^{M}(x_i-y_i)^2
\end{equation}

\begin{equation}\label{eq:ceb}
    Loss_{CEB} = \begin{cases}
        1 - \cos(x_1, x_2),      &\text{if $y = +1$}\\
        \max(0, \cos(x_1, x_2)), &\text{if $y = -1$}
    \end{cases}
\end{equation}

\begin{equation}\label{eq:f}
    F_1 = 2 \frac{precision \cdot recall}{precision + recall} = \frac{2 TP}{2TP + FP + FN}
 \end{equation}



\section{Experiments}\label{sec:experiments}

\subsection{Environmental Setup}
For our Experiments we are bounded by the Enviroments given from Kaggle \cite{noauthor_kaggle_nodate} and Colab \cite{noauthor_google_nodate}, 
which makes quite complex to find the optimal hyperparameters. Also there is no possibility to Cache or Precalculate all the Datapoints of
Training and Validation without running in to RAM issues. We decided to shift the bottelnecks either in parallel loading the
Data and/or running in exceeding GPU RAM due to the amount of parameters.

\vspace{0.3cm}

\begin{minipage}{0.5\textwidth}
    Colab's Environment:
    \begin{itemize}
        \item CPU: Intel(R) Xeon(R) @ 2.00GHz
        \item Number of available Cores: 2
        \item System Ram: 12 GB
        \item GPU: Nvidia Tesla T4
        \item GPU Ram: 15 GB
        \item Time limit: 3-4 Hours a Day/Session
    \end{itemize}
\end{minipage}
\begin{minipage}{0.5\textwidth}
    Kaggle's Environment:
    \begin{itemize}
        \item CPU: Intel(R) Xeon(R) @ 2.00GHz
        \item Number of available Cores: 4
        \item System Ram: 32 GB
        \item GPU: Nvidia Tesla P100
        \item GPU Ram: 16 GB
        \item Time limit: 30 Hour a Week with 9h each Session
    \end{itemize}
\end{minipage}

\vspace{0.3cm}

The general handling of the loops is based on Pytorch \cite{noauthor_pytorch_nodate}, Lightning AI \cite{noauthor_lightning_nodate}
implementation and for loading the pretrained BERT Model we are using Huggingface (transfromer libary) \cite{noauthor_hugging_2023}.
Therefore, we can focus the Experiments on implementing strategies to enhance BERT's overall performance tested on four different Seeds.











\newpage
\subsection{Learning Rate}
Learning Rate, Token Length and Mixed-Precision

\begin{table}[h]
    \centering
    \caption{F1 Values of the Baseline BertModelForSentenceClassification of the Last (50th) Epoch}
    \label{tab:my-table}
    \begin{tabular}{|c||cccc|c|}
    \hline
    \multicolumn{1}{|l||}{\multirow{2}{*}{Learning Rate}} & \multicolumn{4}{c|}{Seed}                                                                    & \multicolumn{1}{c|}{\multirow{2}{*}{mean $\pm$ std}} \\ \cline{2-5}
    \multicolumn{1}{|l||}{}                               & \multicolumn{1}{c|}{0}     & \multicolumn{1}{c|}{42}    & \multicolumn{1}{c|}{1998}  & 1M    & \multicolumn{1}{c|}{}                                \\ \hline\hline
    3e-6                                                 & \multicolumn{1}{c|}{0.637} & \multicolumn{1}{c|}{0.554} & \multicolumn{1}{c|}{0.663} & 0.652 & 0.614 $\pm$ 0.034                           \\ \hline
    4e-6                                                 & \multicolumn{1}{c|}{0.589} & \multicolumn{1}{c|}{0.614} & \multicolumn{1}{c|}{0.649} & 0.615 & 0.616 $\pm$ 0.019                                    \\ \hline
    5e-6                                                 & \multicolumn{1}{c|}{0.636} & \multicolumn{1}{c|}{0.640} & \multicolumn{1}{c|}{0.638} & 0.657 & 0.630 $\pm$ 0.026                                    \\ \hline
    6e-6                                                 & \multicolumn{1}{c|}{0.661} & \multicolumn{1}{c|}{0.654} & \multicolumn{1}{c|}{0.602} & 0.647 & 0.640 $\pm$ 0.021                                    \\ \hline
    7e-6                                                 & \multicolumn{1}{c|}{0.644} & \multicolumn{1}{c|}{0.578} & \multicolumn{1}{c|}{0.630} & 0.667 & 0.621 $\pm$ 0.040                                    \\ \hline
    \end{tabular}
\end{table}












\newpage
\subsection{Permutation}

\begin{equation}
    \resizebox{0.91\hsize}{!}{%
     $Sentence = [CLS] \; \textcolor{red}{stm}\;
                 [SEP] \; \textcolor{green}{sec}\;
                 [SEP] \; \textcolor{blue}{\pi(p_1), \pi(p_2), ... }\;
                 [SEP] \; \textcolor{blue}{\pi(s_1), \pi(s_2), ... }\;
                 [SEP]$      
    }
\end{equation}

\begin{table}[h]
    \centering
    \caption{F1 Values of the Permutation trial on the Baseline Model from the last (50th) Epoch}
    \label{tab:my-table}
    \begin{tabular}{|c|c||cccc|c|}
    \hline
    \multirow{2}{*}{Learning Rate} & \multirow{2}{*}{Permutation} & \multicolumn{4}{c|}{Seed}                                                                    & \multirow{2}{*}{mean $\pm$ std} \\ \cline{3-6}
                                   &                              & \multicolumn{1}{c|}{0}     & \multicolumn{1}{c|}{42}    & \multicolumn{1}{c|}{1998}  & 1M    &                                 \\ \hline \hline
    \multirow{2}{*}{5e-6}          & No                           & \multicolumn{1}{c|}{0.636} & \multicolumn{1}{c|}{0.640} & \multicolumn{1}{c|}{0.638} & 0.657 & 0.630 $\pm$ 0.026               \\ \cline{2-7} 
                                   & Yes                          & \multicolumn{1}{c|}{0.633} & \multicolumn{1}{c|}{0.652} & \multicolumn{1}{c|}{0.620} & 0.643 & \textbf{0.643 $\pm$ 0.016}      \\ \hline
    \end{tabular}
\end{table}














\newpage
\subsection{Dataset Expansion}\label{ch:ds}

\begin{table}[h]
    \centering
    \caption{F1 Values of the Dataset Test of Last (50th) Epoch}
    \label{tab:my-table}
    \begin{tabular}{|c|c||cccc|c|}
    \hline
    \multirow{2}{*}{Dataset + SemEval} & \multirow{2}{*}{Strategy} & \multicolumn{4}{c|}{Seed}                                                                    & \multirow{2}{*}{mean $\pm$ std} \\ \cline{3-6}
                                       &                           & \multicolumn{1}{c|}{0}     & \multicolumn{1}{c|}{42}    & \multicolumn{1}{c|}{1998}  & 1M    &                                 \\ \hline\hline
    Baseline                           &                           & \multicolumn{1}{c|}{0.636} & \multicolumn{1}{c|}{0.640} & \multicolumn{1}{c|}{0.638} & 0.657 & 0.630 $\pm$ 0.026               \\ \hline\hline
    \multirow{2}{*}{SNLI}              & CombDL                    & \multicolumn{1}{c|}{0.597} & \multicolumn{1}{c|}{0.613} & \multicolumn{1}{c|}{0.663} & 0.573 & 0.617 $\pm$ 0.042               \\ \cline{2-7} 
                                       & CombDS                    & \multicolumn{1}{c|}{0.619} & \multicolumn{1}{c|}{0.669} & \multicolumn{1}{c|}{0.637} & 0.664 & 0.643 $\pm$ 0.019               \\ \hline
    \multirow{2}{*}{HEALTHVER}         & CombDL                    & \multicolumn{1}{c|}{0.595} & \multicolumn{1}{c|}{0.637} & \multicolumn{1}{c|}{0.602} & 0.522 & 0.620 $\pm$ 0.009               \\ \cline{2-7} 
                                       & CombDS                    & \multicolumn{1}{c|}{0.636} & \multicolumn{1}{c|}{0.621} & \multicolumn{1}{c|}{0.631} & 0.622 & 0.635 $\pm$ 0.017               \\ \hline
    \multirow{2}{*}{SCIFACT}           & CombDL                    & \multicolumn{1}{c|}{0.586} & \multicolumn{1}{c|}{0.547} & \multicolumn{1}{c|}{0.602} & 0.561 & 0.574 $\pm$ 0.022               \\ \cline{2-7} 
                                       & CombDS                    & \multicolumn{1}{c|}{0.687} & \multicolumn{1}{c|}{0.619} & \multicolumn{1}{c|}{0.664} & 0.645 & 0.654 $\pm$ 0.025               \\ \hline\hline
    
    \multirow{2}{*}{SCIFACT,HEALTHVER} & CombDL                    & \multicolumn{1}{c|}{0.580} & \multicolumn{1}{c|}{0.564} & \multicolumn{1}{c|}{0.551} & 0.570 & 0.566 $\pm$ 0.010               \\ \cline{2-7} 
                                       & CombDS                    & \multicolumn{1}{c|}{0.664} & \multicolumn{1}{c|}{0.637} & \multicolumn{1}{c|}{0.657} & 0.649 & 0.652 $\pm$ 0.010               \\ \hline 
    \multirow{2}{*}{SCIFACT,SNLI}      & CombDL                    & \multicolumn{1}{c|}{0.645} & \multicolumn{1}{c|}{0.607} & \multicolumn{1}{c|}{0.545} & 0.567 & 0.591 $\pm$ 0.039               \\ \cline{2-7} 
                                       & CombDS                    & \multicolumn{1}{c|}{0.658} & \multicolumn{1}{c|}{0.574} & \multicolumn{1}{c|}{0.624} & 0.643 & 0.625 $\pm$ 0.032               \\ \hline 
    \multirow{2}{*}{HEALTHVER,SNLI}    & CombDL                    & \multicolumn{1}{c|}{0.} & \multicolumn{1}{c|}{0.} & \multicolumn{1}{c|}{0.} & 0. & 0. $\pm$ 0.              \\ \cline{2-7} 
                                       & CombDS                    & \multicolumn{1}{c|}{0.} & \multicolumn{1}{c|}{0.} & \multicolumn{1}{c|}{0.} & 0. & 0. $\pm$ 0.     \\ \hline 
    
    \end{tabular}
\end{table}










\newpage
\subsection{Architecture}

\begin{table}[h]
    \centering
    \caption{F1 Values of the Architecture Test from the Last (50th) Epoch}
    \label{tab:my-table}
    \begin{tabular}{|c|c||cccc|c|}
    \hline
    \multirow{2}{*}{Model} & \multirow{2}{*}{Desciption}                                                                       & \multicolumn{4}{c|}{Seed}                                                                    & \multirow{2}{*}{mean $\pm$ std} \\ \cline{3-6}
                           &                                                                                                   & \multicolumn{1}{c|}{0}     & \multicolumn{1}{c|}{42}    & \multicolumn{1}{c|}{1998}  & 1M    &                                 \\ \hline\hline
    Baseline               & \begin{tabular}[c]{@{}c@{}}BertModelForSequenceClassification\\ and CrossEntropyLoss\end{tabular} & \multicolumn{1}{c|}{0.636} & \multicolumn{1}{c|}{0.640} & \multicolumn{1}{c|}{0.638} & 0.657 & 0.630 $\pm$ 0.026               \\ \hline
    v2                     & \begin{tabular}[c]{@{}c@{}}(u, v, x, y) as input to FFN\\ and CrossEntropyLoss\end{tabular}       & \multicolumn{1}{c|}{0.652} & \multicolumn{1}{c|}{0.619} & \multicolumn{1}{c|}{0.629} & 0.664 & 0.635 $\pm$ 0.024               \\ \hline
    v3                     & \begin{tabular}[c]{@{}c@{}}(u, v) as input to FNN\\ and CrossEntropyLoss\end{tabular}             & \multicolumn{1}{c|}{0.708} & \multicolumn{1}{c|}{0.583} & \multicolumn{1}{c|}{0.573} & 0.611 & 0.619 $\pm$ 0.054               \\ \hline
    v4                     & \begin{tabular}[c]{@{}c@{}}(u, v, |u-v|) as input to FNN\\ and CrossEntropyLoss\end{tabular}      & \multicolumn{1}{c|}{0.638} & \multicolumn{1}{c|}{0.631} & \multicolumn{1}{c|}{0.618} & 0.640 & 0.632 $\pm$ 0.009               \\ \hline
    v5                     & \begin{tabular}[c]{@{}c@{}}CosSim(u, v)\\ and CosineEmbeddingLoss\end{tabular}                    & \multicolumn{1}{c|}{0.614} & \multicolumn{1}{c|}{0.632} & \multicolumn{1}{c|}{0.637} & 0.664 & 0.630 $\pm$ 0.023               \\ \hline
    v6                     & \begin{tabular}[c]{@{}c@{}}CosSim(u, v)\\ and MSELoss\end{tabular}                                & \multicolumn{1}{c|}{0.667} & \multicolumn{1}{c|}{0.654} & \multicolumn{1}{c|}{0.667} & 0.646 & 0.664 $\pm$ 0.011               \\ \hline
    \end{tabular}
\end{table}











\newpage
\subsection{Adapter}


\begin{table}[h]
    \centering
    \caption{F1 Values of the Adapter Trial from the Last (50th) Epoch}
    \label{tab:my-table}
    \begin{tabular}{|l||cccc|c|}
    \hline
    \multicolumn{1}{|c||}{\multirow{2}{*}{Model}} & \multicolumn{4}{c|}{Seed}                                                                    & \multirow{2}{*}{mean $\pm$ std} \\ \cline{2-5}
    \multicolumn{1}{|c||}{}                       & \multicolumn{1}{c|}{0}     & \multicolumn{1}{c|}{42}    & \multicolumn{1}{c|}{1998}  & 1M    &                                 \\ \hline\hline
    Baseline BERT                                 & \multicolumn{1}{c|}{0.636} & \multicolumn{1}{c|}{0.640} & \multicolumn{1}{c|}{0.638} & 0.657 & 0.630 $\pm$ 0.026               \\ \hline
    Adapter BERT                                  & \multicolumn{1}{c|}{0.619} & \multicolumn{1}{c|}{0.587} & \multicolumn{1}{c|}{0.602} & 0.626 & 0.608 $\pm$ 0.015               \\ \hline
    Baseline SIMCSE                               & \multicolumn{1}{c|}{0.598} & \multicolumn{1}{c|}{0.667} & \multicolumn{1}{c|}{0.696} & 0.640 & 0.650 $\pm$ 0.036               \\ \hline
    Adapter SIMCSE                                & \multicolumn{1}{c|}{0.643} & \multicolumn{1}{c|}{0.673} & \multicolumn{1}{c|}{0.615} & 0.654 & 0.646 $\pm$ 0.021      \\ \hline
    \end{tabular}
\end{table}


\section{Conclution}

In conclusion, this paper presents a thorough investigation into text entailment classification, 
focusing on Clinical Trial Reports (CTR) annotation. Through careful dataset curation and 
experimentation using the SemEval task 2 dataset, the study explores various approaches, 
including BERT-based models and adapter tuning. Despite challenges in environmental setup, 
the findings highlight the importance of learning rate optimization, dataset expansion 
strategies, and sentence embedding architectures in influencing model performance. 
The paper contributes valuable insights to natural language processing research, 
particularly in clinical text analysis, and sets the stage for further exploration 
in related domains.
%\newpage
\endgroup

%
% ---- Bibliography ----
%
% BibTeX users should specify bibliography style 'splncs04'.
% References will then be sorted and formatted in the correct style.
%
\bibliographystyle{splncs04}
\bibliography{PrakNLP} %Filename

\end{document}
